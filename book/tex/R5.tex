\chapter{Reference classes}

R has three object oriented (OO) systems: {[}{[}S3{]}{]}, {[}{[}S4{]}{]}
and Reference Classes (where the latter were for a while referred to as
{[}{[}R5{]}{]}, yet their official name is Reference Classes). This page
describes this new reference-based class system.

Reference Classes (or refclasses) are new in R 2.12. They fill a long
standing need for mutable objects that had previously been filled by
non-core packages like \texttt{R.oo}, \texttt{proto} and
\texttt{mutatr}. While the core functionality is solid, reference
classes are still under active development and some details will change.
The most up-to-date documentation for Reference Classes can always be
found in \texttt{?ReferenceClasses}.

There are two main differences between reference classes and S3 and S4:

\begin{itemize}
\itemsep1pt\parskip0pt\parsep0pt
\item
  Refclass objects use message-passing OO
\item
  Refclass objects are mutable: the usual R copy on modify semantics do
  not apply
\end{itemize}

These properties makes this object system behave much more like Java and
C\#. Surprisingly, the implementation of reference classes is almost
entirely in R code - they are a combination of S4 methods and
environments. This is a testament to the flexibility of S4.

Particularly suited for: simulations where you're modelling complex
state, GUIs.

Note that when using reference based classes we want to minimise side
effects, and use them only where mutable state is absolutely required.
The majority of functions should still be ``functional'', and side
effect free. This makes code easier to reason about (because you don't
need to worry about methods changing things in surprising ways), and
easier for other R programmers to understand.

Limitations: can't use enclosing environment - because that's used for
the object.

\section{Classes and instances}

Creating a new reference based class is straightforward: you use
\texttt{setRefClass}. Unlike \texttt{setClass} from S4, you want to keep
the results of that function around, because that's what you use to
create new objects of that type:

\begin{verbatim}
# Or keep reference to class around.
Person <- setRefClass("Person")
Person$new()
\end{verbatim}

A reference class has three main components, given by three arguments to
\texttt{setRefClass}:

\begin{itemize}
\item
  \texttt{contains}, the classes which the class inherits from. These
  should be other reference class objects:

\begin{verbatim}
setRefClass("Polygon")
setRefClass("Regular")

# Specify parent classes
setRefClass("Triangle", contains = "Polygon")
setRefClass("EquilateralTriangle", 
  contains = c("Triangle", "Regular"))
\end{verbatim}
\item
  \texttt{fields} are the equivalent of slots in \texttt{S4}. They can
  be specified as a vector of field names, or a named list of field
  types:

\begin{verbatim}
setRefClass("Polygon", fields = c("sides"))
setRefClass("Polygon", fields = list(sides = "numeric"))
\end{verbatim}
\end{itemize}

The most important property of refclass objects is that they are
mutable, or equivalently they have reference semantics:

\begin{verbatim}
    Polygon <- setRefClass("Polygon", fields = c("sides"))
    square <- Polygon$new(sides = 4)
    
    triangle <- square
    triangle$sides <- 3
    
    square$sides        
\end{verbatim}

\begin{itemize}
\item
  \texttt{methods} are functions that operate within the context of the
  object and can modify its fields. These can also be added after object
  creation, as described below.

\begin{verbatim}
setRefClass("Dist")
setRefClass("DistUniform", c("a", "b"), "Dist", methods = list(
  mean <- function() {
    (a + b) / 2
  }
))
\end{verbatim}
\end{itemize}

You can also add methods after creation:

\begin{verbatim}
# Instead of creating a class all at once:
Person <- setRefClass("Person", methods = list(
  say_hello = function() message("Hi!")
))

# You can build it up piece-by-piece
Person <- setRefClass("Person")
Person$methods(say_hello = function() message("Hi!"))
\end{verbatim}

It's not currently possible to modify fields because adding fields would
invalidate existing objects that didn't have those fields.

The object returned by \texttt{setRefClass} (or retrieved later by
\texttt{getRefClass}) is called a generator object. It has methods:

\begin{itemize}
\item
  \texttt{new} for creating new objects of that class. The \texttt{new}
  method takes named arguments specifying initial values for the fields
\item
  \texttt{methods} for modifying existing or adding new methods
\item
  \texttt{help} for getting help about methods
\item
  \texttt{fields} to get a list of fields defined for class
\item
  \texttt{lock} locks the named fields so that their value can only be
  set once
\item
  \texttt{accessors} a convenience method that automatically sets up
  accessors of the form \texttt{getXXX} and \texttt{setXXX}.
\end{itemize}

\section{Methods}

Refclass methods are associated with objects, not with functions, and
are called using the special syntax
\texttt{obj\$method(arg1, arg2, ...)}. (You might recall we've seen this
construction before when we called functions stored in a named list).
Methods are also special because they can modify fields. This is
different

We've also seen this construct before, when we used closures to create
mutable state. Reference classes work in a similar manner but give us
some extra functionality:

\begin{itemize}
\itemsep1pt\parskip0pt\parsep0pt
\item
  inheritance
\item
  a way of documenting methods
\item
  a way of specifying fields and their types
\end{itemize}

Modify fields with \texttt{\textless{}\textless{}-}. Will call accessor
functions if defined.

Special fields: \texttt{.self} (Don't use fields with names starting
with \texttt{.} as these may be used for special purposes in future
versions.)

\texttt{initialize}

\subsection{Common methods}

Because all refclass classes inherit from the same superclass,
\texttt{envRefClass}, they a have common set of methods:

\begin{itemize}
\item
  \texttt{obj\$callSuper}:
\item
  \texttt{obj\$copy}: creates a copy of the current object. This is
  necessary because Reference Classes classes don't behave like most R
  objects, which are copied on assignment or modification.
\item
  \texttt{obj\$field}: named access to fields. Equivalent to
  \texttt{slots} for S4. \texttt{obj\$field("xxx")} the same as
  \texttt{obj\$xxx}. \texttt{obj\$field("xxx", 5)} the same as
  \texttt{obj\$xxx \textless{}- 5}
\item
  \texttt{obj\$import(x)} coerces into this object, and
  \texttt{obj\$export(Class)} coerces a copy of obj into that class.
  These should be super classes.
\item
  \texttt{obj\$initFields}
\end{itemize}

\subsection{Documentation}

Python style doc-strings. \texttt{obj\$help()}.

\section{In packages}

Note: collation Note: namespaces and exporting

\section{In the wild}

Rook package. Scales package?
